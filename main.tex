\documentclass[conference]{IEEEtran}
\IEEEoverridecommandlockouts
% The preceding line is only needed to identify funding in the first footnote. If that is unneeded, please comment it out.
\usepackage{cite}
\usepackage{amsmath,amssymb,amsfonts}
\usepackage{algorithmic}
\usepackage{graphicx}
\usepackage{textcomp}
\usepackage{xcolor}
\usepackage[utf8]{inputenc}
\usepackage[nolist,nohyperlinks]{acronym}
\usepackage[hidelinks]{hyperref}

\def\BibTeX{{\rm B\kern-.05em{\sc i\kern-.025em b}\kern-.08em
    T\kern-.1667em\lower.7ex\hbox{E}\kern-.125emX}}
\begin{document}

\begin{acronym}
    \acro{MFCC}{Mel-frequency Cepstral Coefficients}
    \acro{DTW}{Dynamic Time Warping}
    \acro{STFT}{Short-Time Fourier Transform}
    \acro{ASR}{Automatic Speech Recognition}
    \acro{DCT}{Discrete Cosine Transform}
\end{acronym}

\title{Automatic Speech Recognition\\}
\author{\IEEEauthorblockN{João Borges}
\IEEEauthorblockA{\textit{Telecommunications, Automation and Electronics}\\ 
\textit{Research and Development Center (LASSE)}\\
\textit{Federal University of Pará}\\
Belém, Brazil \\
joao.tavares.borges@itec.ufpa.br}
}

\maketitle

\begin{abstract}
In this work, the Mel-frequency Cepstral Coefficient (MFCC) and 
Dynamic Time Warping (DTW) algorithms are implemented in 
Python to develop an isolated word recognition pipeline.
The system currently shows 60\% accuracy in word identification when 
a built-in MFCC function is used, and 40\% when an
own implementation is utilized instead.
\end{abstract}

\begin{IEEEkeywords}
MFCC, DTW, Speech Recognition 
\end{IEEEkeywords}

\section{Introduction}
The use of \ac{MFCC} to extract audio feature and \ac{DTW} to compare the obtained
values is a classical method of \ac{ASR}. This method leverages the Mel frequency 
scale, that is designed to be a perceptually relevant scale for pitch, meaning 
that equal frequency distances have the same perceptual difference from the point 
of view of a human listener. 

\section{Data Processing Pipeline}
This work uses a dataset composed of 105,000 WAVE audio files, with 30 different 
classes, obtained from Tensorflow\footnote{https://storage.cloud.google.com/download.tensorflow.org/data/\\
speech\_commands\_v0.02.tar.gz}, from which a subset of 5 classes will be used in 
the experiments. The dataset is divided into 80\% for training and 20\% for 
validation. The remaining process can be split into 3 steps:

\begin{itemize}
    \item Calculate \ac{MFCC} for all signals
    \item Perform \ac{DTW} between sample and reference signals 
    \item Obtain the mean of the \ac{DTW} values, which is the recognition rate
\end{itemize}

In the following subsections each of these steps will be described in details.

\subsection{Obtaining the MFCCs}
 The \acp{MFCC} of all audio files are calculated with the following steps:
% \begin{itemize}
% \item The signal goes through a pre-emphasis filter
% \item After that, it gets sliced into overlapping frames
% \item A window function is applied to the frames
% \item A \ac{STFT} is executed and the power spectrums are calculated 
% \item The filter banks are computed
% \item The MFCC is obtained by applying \ac{DCT} to the filter banks
% \item Normalization step
% \end{itemize}

\begin{itemize}
    \item The signal goes through a process of \ac{STFT}
    \item After that, the Mel filter banks are calculated
    \item Then, a dot product between the previous results, 
    followed by a dB conversion, are performed to obtain the 
    melspectrogram
    \item Finally, the \ac{MFCC} is obtained by applying \ac{DCT} to the 
    melspectrogram
\end{itemize}

\subsubsection{Short-Time Fourier Transform Step}

Due to the highly non stationary nature of speech data, composed of different 
phonemes with their own frequencies distributed along the time, it is necessary
to be capable of visualizing how the signal spectrum changes over time, a feature 
that is lacking in the traditional Fourier transform. This motivates the method 
known as \ac{STFT}, that divides the signal into frames through a process of 
windowing and then applies the Fourier transform in each one of them. These 
frames usually need to be overlapped in order to avoid loss of signal.
In the current implementation, the \ac{STFT} adopts the widely used 
\textit{Hann} window for the windowing process.

\subsubsection{Mel filter banks}
To calculate the Mel filter banks, first is necessary to determine the number of 
Mel bands that will be used, then obtain the lowest and highest frequencies of 
the signal in the Mel scale. After that the interval between the lowest and the 
highest frequencies must be divided into a number of points equal to the number 
of Mel bands, evenly separated. These points are then converted back to the Hertz 
scale and rounded to the nearest bin. Finally, the triangular filters are created [Improve]
\subsubsection{The melspectrogram and the MFCC}
In order to obtain the melspectrogram first we need the spectrogram, which is obtained
by the magnitude squared of the \ac{STFT} result. After that, a dot product is executed between
the spectrogram and the Mel filter banks, followed by a power to dB conversion, resulting in 
the melspectrogram. To 
\subsection{Performing the DTW}

\subsection{Evaluating the system performance}
In this step, a process of cross-validation is implemented. \textit{Leave P out} 
cross-validation is [...]
a The code is available publicly\footnote{https://github.com/joaotavares43/asr-jupyter}

\begin{figure}[htbp]
%\centerline{\includegraphics{fig1.png}}
\caption{Example of a figure caption.}
\label{fig}
\end{figure}


% \section*{References}

% Please number citations consecutively within brackets \cite{IEEEhowto:IEEEtranpage}. The 
% sentence punctuation follows the bracket \cite{b2}. Refer simply to the reference 
% number, as in \cite{b3}---do not use ``Ref. \cite{b3}'' or ``reference \cite{b3}'' except at 
% the beginning of a sentence: ``Reference \cite{b3} was the first $\ldots$''

% Number footnotes separately in superscripts. Place the actual footnote at 
% the bottom of the column in which it was cited. Do not put footnotes in the 
% abstract or reference list. Use letters for table footnotes.

% Unless there are six authors or more give all authors' names; do not use 
% ``et al.''. Papers that have not been published, even if they have been 
% submitted for publication, should be cited as ``unpublished'' \cite{b4}. Papers 
% that have been accepted for publication should be cited as ``in press'' \cite{b5}. 
% Capitalize only the first word in a paper title, except for proper nouns and 
% element symbols.

% For papers published in translation journals, please give the English 
% citation first, followed by the original foreign-language citation \cite{b6}.

% \bibliographystyle{./bibliography/IEEEtran}
% \bibliography{./bibliography/IEEEabrv,./bibliography/IEEEexample}

% \vspace{12pt}
% \color{red}
% IEEE conference templates contain guidance text for composing and formatting conference papers. Please ensure that all template text is removed from your conference paper prior to submission to the conference. Failure to remove the template text from your paper may result in your paper not being published.

\end{document}
